\documentclass{article}
%\usepackage[margin=2in]{geometry}
\usepackage[osf,p]{libertinus}
\usepackage{microtype}
\usepackage[pdfusetitle,hidelinks]{hyperref}

\usepackage[series={A,B,C}]{reledmac}
\usepackage{reledpar}

\usepackage{graphicx}
\usepackage{polyglossia}
\setmainlanguage{english}
\setotherlanguage{hebrew}
\gappto\captionshebrew{\renewcommand\chaptername{קאַפּיטל}}
\usepackage{metalogo}


%%linenumincrement*{1}
%%\firstlinenum*{15}
%%\setlength{\Lcolwidth}{0.44\textwidth}
%%\setlength{\Rcolwidth}{0.44\textwidth}

\begin{document}
%%\maxhnotesA{0.8\textheight}
\renewcommand{\abstractname}{\vspace{-\baselineskip}}
\title{Diary of the Vilna Ghetto}
\author{Herman Kruk  \\ Transl. Ilan Pillemer}
\date{\today}

\maketitle

\tableofcontents
\newpage

\begin{pairs}

\begin{Rightside}

\begin{RTL}
\begin{hebrew}
\firstlinenumR{10000000}
\beginnumbering
%%\numberpstarttrue

\autopar


%%
װעגען דער געשיכטע פֿון הערמאַן קרוקס „טאָגבוך פֿון װילנער געטאָ“, ספּעציעל װעגן די גילגולים פֿון דער דאָזיקער זעלטענער חורבן־כראָניק,
דערצײלט פּנחס שװאַרץ אין דער ביאָגראַפֿיאַ פֿון הערמאַן קרוקן װאָס מיר גיבן דאָ װײַטער.
דאָ װיל איך בלויז אָפּשטעלן אויף אַ רײ פּרטים פֿון טעכניש־רעדאַקציאָנעלן כאַראַקטער:

\endnumbering
\end{hebrew}
\end{RTL}
\end{Rightside}


\begin{Leftside}
\begin{english}
\section{
Introduction - Mordecai W. Bernstein \\  \RLE{
הקדמה ־ מרדכי װ. בערנשטײן
}.  }
\beginnumbering
\autopar



 
The story of Herman Kruk's \emph{Diary of the Vilna Ghetto}, specifically the reincarnation\footnoteA{
The term \RLE{גילגולים}
can also refer in biblical exegesis to the passage of the bones (literally tumbling over and over) of the deceased at the end of times through underground passages
to the Holy Land, where during the Messianic era the body will live again. Thus this Loshen Kodesh word, with its biblical overtones, captures the travails the book went through 
after being buried to finally being reconstituted and published.} 
of the aforementioned rare hurban-chronicle,
will be told by Pinkhos Schwartz in the biography of Herman Kruk which is provided further on. 


\endnumbering
\end{english}
\end{Leftside}

\end{pairs}
\Columns


\end{document}



















































