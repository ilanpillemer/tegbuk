\documentclass{article}
%\usepackage[margin=2in]{geometry}
\usepackage[osf,p]{libertinus}
\usepackage{microtype}
\usepackage[pdfusetitle,hidelinks]{hyperref}

\usepackage[series={A,B,C}]{reledmac}
\usepackage{reledpar}

\usepackage{graphicx}
\usepackage{polyglossia}
\setmainlanguage{english}
\setotherlanguage{hebrew}
\gappto\captionshebrew{\renewcommand\chaptername{קאַפּיטל}}
\usepackage{metalogo}


%%linenumincrement*{1}
%%\firstlinenum*{15}
%%\setlength{\Lcolwidth}{0.44\textwidth}
%%\setlength{\Rcolwidth}{0.44\textwidth}

\begin{document}
%%\maxhnotesA{0.8\textheight}
\renewcommand{\abstractname}{\vspace{-\baselineskip}}
\title{Diary of the Vilna Ghetto}
\author{Herman Kruk  \\ Transl. Ilan Pillemer}
\date{\today}

\maketitle

\tableofcontents
\newpage

\begin{pairs}

\begin{Rightside}

\begin{RTL}
\begin{hebrew}
\firstlinenumR{10000000}
\beginnumbering
%%\numberpstarttrue

\autopar

װעגען דער געשיכטע פֿון הערמאַן קרוקס „טאָגבוך פֿון װילנער געטאָ“, ספּעציעל װעגן די גילגולים פֿון דער דאָזיקער זעלטענער חורבן־כראָניק,
דערצײלט פּנחס שװאַרץ אין דער ביאָגראַפֿיאַ פֿון הערמאַן קרוקן װאָס מיר גיבן דאָ װײַטער.
דאָ װיל איך בלויז אָפּשטעלן אויף אַ רײ פּרטים פֿון טעכניש־רעדאַקציאָנעלן כאַראַקטער:

א. דאָס לעצע זײַטל פֿון אָריגינאַל (מאַשינשריפגט) טראַגט דעם נומער 757.
אָבער אין מיטן פֿעלן דאָ און דאָרט אַ צאָל בלעטער און אין סך־הכּל האָבן מיר אין אונדזער רשות 510 זײַטן פֿון אָריגינאַל.
תּחילת זענען אין 1948 אָנגעקומען קײן ניו־יאָרק 380 זײַטן, אָבער מיט 11 ײַאָר שפּעטער, אין 1959,
װען מיר האָבן שוין געהאַלטן אין מיטן צוגרײטן דאָס „טעגבאָך“ צום דרוק, האָבן זיך אין אַרכײַװ פֿון יד ושם אין ירושלים אָפּגעפֿונען נאָך 130 זײַטן.

ב. די זײַטן װאָס פֿעלן האָבן בכן היפּשע בלויזן (אין רובֿ פאַלן פֿעלן  אין אַ צאָל ערטער נאָר אײנצלנע זײַטן, מיט פֿאַרצײכענונגען
פֿון אײנעם אָדער צװײ טעג, אָבער אין אַ קלײנער צאָל ערטער פֿאַלן עטלעכע צענדליג זײַטן כּסדר).
אײנע פֿון די װיכטיקסטע עובֿדות בײַם צוגרײטן דאָס „טעגבוך“ צום דרוק איז דעריבער געװען װי װײַט מעגלעך צו שאַפֿן, װי איך װאָלט עס אָנרופֿן, 
רינגען, װאָס זאָלן צונויפֿקײטלען די איבערגעריסענע כראָניק.

די ידיעות צום דערגאַנצן דאָס װאָס עס פֿעלט האָבן מיר גענומען פֿון אַרכיװ־מאַטעריאַלן (דאָקומענטן, גבֿית־עדותן וכדומה), פֿון אַ רײ שוין פֿאַרעפֿנטלעכטע
אַרבעטן װעגען װילנע און פֿון פּערזאָנען װאָס זענען אין יעדער צײַט געװען אין
װילנער געטאָ. בײַ יעדען פֿאַל האָבן מיר אָנגעװיזן דעם מקור.

די דאָזיקה אזוי גערופֿענע רינגען האָבן מיר געגעבן אויף די שײכדיקה ערטער מיט קלענער
שריפֿט און אין קאַנטיקה קלאַמערן, כּדי זײ זאָלן זיך בולט
אונטערשײדן פֿון קרוקס אָריגינעלן טעקסט. עס װערט דערבײַ אויך אָנגעגעבן
װיפֿל זײַטן פֿון אָריגינאַל עס פֿעלן אויף דעם געגעבענעם אָרט.

ג. דאָס „טעגבאָך“ נעמט אַרום אַ פּעריאָד פֿון פֿולע צװײ יאָר װילנע אונטער
דער האַצישער אָקופּאַציע. די ערשטע דאַטע אין „טאָגביך“ איז דער 23סטער יוני
1941, די לעצטע - דער 14טער יולי 1943. אָבער דאָס איז זיכער נישט דער
סוף פֿון „טאָגבוך“. קודם־כּל פֿאַרענדיקט זיך די לעצטע זײַט אין מיטן פֿון אַ זאַץ,
און עס זענען פֿאַראַן ידיעות אַז הערמאַן קרוק האָט נאָך שפּעטער, סײַ אין
געטאָ סײַ װען ער שוין געװען אין אַ לאַגער אין עסטלאַנד, װײַטער געפֿירט
זײַן כראָניק און פֿאַרצײכנס די געשעענישן. אָבער דער סוף טאָגבוך איז צו
אונדז נישט דערגאַנגען

ד. עס איז נײטיק געװען צוצוגעבן הערות און אויסטײַשונגען פֿון אַ פֿאַרשײדענעם כאַראַקטער, למשל:


 (1) צוליב קאָנספּיראַטיװע טעמים האָט קרוק גאָר אָפֿט בײַם פֿאַרצײכענען
 נעמען אָנגעגעבן בלויז איניציאַלן אָדער נאָר דעם ערשטן נאָמען
 אָדער דעם פֿאַמיליע־נאָמען. האָט מען דאָס בכן געדאַרפֿט דעשיפֿרירן
 און דערגאַנצן. דאָס זעלביקע בנוגע אַ רײ אָרגאַניזאַציעס און אינסטיטוציעס װאָס זענען
 פֿאַרצײכנט מיט ראָשי־תּיבֿות אָדער קיצורים, װאָס
 זענען דעמאָלט, בימי געטאָס און לאַגערן, געװען באַקאַנט, אָבער הײַנט
 דאַרף מען זײ אויסטײַטשן. אַלע אַזעלכע דערגאַנצונגען האָבן מיר געגעבן
 אין קאַנטיקע קלאַמערן.

(2)
צו אַ סך נעמען האָבן מיר (אין באַזונדערע הערות) צוגעגעבן
ביאָגראַפֿישׂע פּרטים און װוּ מעגלעך אויך װעגן דעם װײַטערדיקן גורל
פֿון דער דערמאָנטער פּערזאָן, ספּעציעל װעגן עס רעדט  זיך װעגן מענטשן
װאָס האָבן פֿאַרנומען אַ געזעלשאַפֿטלעכע פּאָזיציע.

(3)
פֿון טײל זײַטן זענען אָפּגעריסן גאַנצע שטיקער, אָדער עס פֿאַלן
טײלן פֿון שורות (זען רעפּראָדוקציעס, זז' 270, 447), איז אויב מיר האָבן
זיך געקענט משער זײַן מיט אַ געװיסער מאָס זיצערקײט װאָס דאָרטן פֿעלט
האָבן מיר די דאָזיקע שטעלן דערגאַנצט און די דערגאַנצונגען אַרײַנגענומען
אין קאַנטיקע קלאַמערן. אויב מיר האָבן נישט געקענט עפּעס צוגעבן
האָבן מיר געשטעלט פּינטעלעך.

(4)
אין אַ צאָל ערטער, װוּ מיר האָבן באַמערקט אַ בפֿירושן טעות
אין טעקסט (למשל אין אַ דאַטע, אין אַ נאָמען אד“גל) האָבן מיר דאָס
אויסגעבעסערט אין אַ הערה.

(5)
גאַנץ אָפֿט באַגלײט קרוק זײַנע טאַג־גאַרצײכענונגען מיטן צוגאָב,
אַז ער לײגט בײַ אַ דאָקומענט, אָבער כּמעט אומטום איז דער
 דערמאָנטער דאָקומענט נישט בײַגעלײגט. טײל פֿון די דאָזיקע 
 דאָקומענט האָבן מיר געפֿונען אין אַרכיװ פֿון ייִװאָ (ס'רובֿ אין דער
  קאַטשערגינסקי־סוצקעװער־קאַלעקציע), אָדער ערגעץ אַנדערש װוּ, טײל זענען
  שוין אָפּגעדרוקט אין אַנדערע װערק. אין אַזעלכע פֿאַלן האָבן מיר אָדער
  רעפּראָדוצירט דעם דאָקומענט, אָדער אָנגעװיזן װוּ ער געפֿינט זיך.

ה. בדרך־כּלל האָבן מיר דער שׂפּראַך פֿוּן „טאָגבוזש“ גאָרנישט געביטן.
מיר האָבן איבערגעלאָזט אַזעלעכע און אויסדרוקן װאָס געװײנטלעך
װאָלט מען זײ אין אַ ייִװאָ־אויסגאַבע נישט געניצט. מיר האָבן אויך דאָ און
דאָרטן איבערגעלאָזט אַ גראַמאַטישן פֿעלער. געביטן האָבן מיר בלויז דעם
אויסלײג און אין געצײלט ערטער צוגעשטלעט אָדער איבערגעשטעלט אַ װאָרט
אויב דער זאַץ איז נישט געװען גענוג קלאָר. מיר האָבן אויך נישט מקפּיד
געװען אויף אײנהײטלעכקײט: אַ מאָל שרײַבט קרוק די געטאָ, און אַ מאָל
 דער געטאָ; אַ מאַל  װוּינונג־אָפּטײל און אַ מאָל װוּינונג־אָפּטײלונג אַד“גל.
 
 ו. מיר האָבן זיך באַמיט נישט צו באַשװערן דאָס בוך מיט צו פֿיל הערות.
 און װוּ עס איז נאָר מעגלעך געװען האָבן מיר געמאַכט די דערגאַנצונגען אין
 טעקסט גופֿה. דער עיקר האָבן מיר דאָס געטאָן בײַ נעמען פֿון פּערזאָנען און
 אינסטיטוציעס, למשל: [יעקבֿ] גערשטײן, [גערשון] פּלודערמאַכער, גרישע [יאשונסקי], [הערש] גוט[געשטאַלט], ג[ענס], „ב[ונד]“, „ר[רויטע“, אד“גל.


\endnumbering
\end{hebrew}
\end{RTL}
\end{Rightside}


\begin{Leftside}
\begin{english}
\section{
Introduction - Mordecai W. Bernstein \\  \RLE{
הקדמה ־ מרדכי װ. בערנשטײן
}.  }
\beginnumbering
\autopar

The story of Herman Kruk's \emph{Diary of the Vilna Ghetto}, specifically the reincarnation\footnoteA{
The term \RLE{גילגולים}
can also refer in biblical exegesis to the passage of the bones (literally tumbling over and over) of the deceased at the end of times through underground passages
to the Holy Land, where during the Messianic era the body will live again. Thus this Loshen Kodesh word, with its biblical overtones, captures the travails the book went through 
after being buried to finally being reconstituted and published.} 
of the aforementioned rare hurban-chronicle,
will be told by Pinkhos Schwartz in the biography of Herman Kruk which is provided further on. 
Here I only want to highlight a list of technical-editorial details.

\emph{a}. The last page of the  (typewritten) original is numbered 757. However here and there in the middle a number of pages are missing,
and in the end we have in our control 510 pages of the original.
At first, in 1948, 380 pages were brought to New York. But, 11 years later in 1959, 
when we were already in the process of preparing the \emph{Diary} for publication, a further 130 pages were discovered in the archives of Yad Vashem in Yerushalayim.

\emph{b}. The missing pages consequently are substantial gaps. (In most cases, in a number of places, only individual pages are missing - which would have recorded one or two days. 
However in a small number of places several tens of pages are missing in order). 
One of the most important issues in preparing the \emph{Diary} for publication is
therefore the consideration of how distant was the possibility to procure, what I like to call, \emph{links};
which would chain together the disconnected chronicle.

The information that supplements that which missing was taken from archive materials 
(documents, witness testimonies and the like), from a series of already published works
about Vilna, and from people which were in the Vilna at that time. In each case we have the
indicated the source.
 
The aforementioned, so called "links", we have provided in the suitable places in a smaller
script and surrounded by square brackets; in order that they should stand out clearly from 
Kruk's original text. 
These are thereby ``standing in the place of" the missing original pages, in the relevant spots themselves.

\emph{c}. The \emph{Diary} covers a period of two full years of Vilna under the Nazi occupation.
The first date in \emph{Diary} is the 23\textsuperscript{rd} June 1941, the last - the 14\textsuperscript{th}
July 1943. However, this is definitely not the end of \emph{Diary}. First of all, the 
last page ends in the middle of a sentence, and there is also evidence that
Herman Kruk continued later, both in the Ghetto and also when he was in a camp in Estonia,
to further manage his chronicle and record what was taking place. But the rest of the diary
did not make its way to us.

\emph{d}. It was necessary to provide annotations or explications of varied character, for example:


(1) On account of conspiracy reasons Kruk often completely tokenised names, substituting
only initials, or just the first name or surname. These need to therefore be deciphered and
supplemented. There is the same concern with a series of organisations and institutions which
are tokenised with acronyms or abbreviations - which were at the time, in the days of the
ghetto and camps, known. However, today, these need to be explicated. All of these supplements we
have given in square brackets.

(2) For many names we have (through separate annotations) provided biographical details
and where possible also the eventual fate of the referenced person, especially when the people
who are being spoken about took up a communal position.

(3) From a few pages whole pieces have been torn off, or sections of lines are missing, (see
facsimiles on pages 270, 447). Therefore if we were able to surmise with a reasonable
degree of confidence what was missing, we provided supplements and surrounded the supplements
with square brackets. If we did not know what to provide we then inserted a set of dots.

(4) In a number of places where we noticed obvious mistakes (for example in a date, in a name, and the similar),
we corrected them in an annotation.

(5) Very often Kruk linked his diary notes to an addendum, such as referring to a document, but
almost all of these mentioned documents aren't provided. Some of these documents we have
found in the YIVO archives (mostly in the Sutzkever Kaczerginski Collection\footnoteA{http://www.yivoarchives.org/index.php?p=collections/findingaid\&id=932702\&q=Sutzkever\&rootcontentid=219617}),
or somewhere else, some were already published in other work. In such cases we either reproduce the document or detail where it can be found.

\emph{e}. Normally we have kept the language of \emph{Diary} as constructed. We have left alone such words
and expressions which normally one would not use in a YIVO publication. We have also left alone here
and there grammatical mistakes. Thus we have only altered the orthography and in rare places added or moved a word
when otherwise the sentence would not be sufficiently clear. We have not been fastidious with uniformity: sometimes Kruk
writes \emph{the} ghetto  with a feminine article and sometimes with a male article; sometimes he writes the word \emph{section}
spelt one way and sometimes spelt another way, and other similar examples. 

\emph{f}. We have made an effort to not burden the book with too many annotations.
And where it is possible we have made the additions in the text body. We have
in particular done this with names from people and institutions. For example:
[Yankel] Gerstein, [Gershon] Pludermacher, Grisha [Yashunski], [Hersh] Gut[gestalt], G[ens],
B[und], R[oiter], and other similar examples.

\endnumbering
\end{english}
\end{Leftside}

\end{pairs}
\Columns


\end{document}



















































