\documentclass{article}
%\usepackage[margin=2in]{geometry}
\usepackage[osf,p]{libertinus}
\usepackage{microtype}
\usepackage[pdfusetitle,hidelinks]{hyperref}

\usepackage[series={A,B,C}]{reledmac}
\usepackage{reledpar}

\usepackage{graphicx}
\usepackage{polyglossia}
\setmainlanguage{english}
\setotherlanguage{hebrew}
\gappto\captionshebrew{\renewcommand\chaptername{קאַפּיטל}}
\usepackage{metalogo}


%%linenumincrement*{1}
%%\firstlinenum*{15}
%%\setlength{\Lcolwidth}{0.44\textwidth}
%%\setlength{\Rcolwidth}{0.44\textwidth}

\begin{document}
%%\maxhnotesA{0.8\textheight}
\renewcommand{\abstractname}{\vspace{-\baselineskip}}
\title{Diary of the Vilna Ghetto}
\author{Herman Kruk  \\ Transl. Ilan Pillemer}
\date{\today}

\maketitle

\tableofcontents
\newpage

\begin{pairs}

\begin{Rightside}

\begin{RTL}
\begin{hebrew}
\firstlinenumR{10000000}
\beginnumbering
%%\numberpstarttrue

\autopar

װעגען דער געשיכטע פֿון הערמאַן קרוקס „טאָגבוך פֿון װילנער געטאָ“, ספּעציעל װעגן די גילגולים פֿון דער דאָזיקער זעלטענער חורבן־כראָניק,
דערצײלט פּנחס שװאַרץ אין דער ביאָגראַפֿיאַ פֿון הערמאַן קרוקן װאָס מיר גיבן דאָ װײַטער.
דאָ װיל איך בלויז אָפּשטעלן אויף אַ רײ פּרטים פֿון טעכניש־רעדאַקציאָנעלן כאַראַקטער:

א. דאָס לעצע זײַטל פֿון אָריגינאַל (מאַשינשריפגט) טראַגט דעם נומער 757.
אָבער אין מיטן פֿעלן דאָ און דאָרט אַ צאָל בלעטער און אין סך־הכּל האָבן מיר אין אונדזער רשות 510 זײַטן פֿון אָריגינאַל.
תּחילת זענען אין 1948 אָנגעקומען קײן ניו־יאָרק 380 זײַטן, אָבער מיט 11 ײַאָר שפּעטער, אין 1959,
װען מיר האָבן שוין געהאַלטן אין מיטן צוגרײטן דאָס „טעגבאָך“ צום דרוק, האָבן זיך אין אַרכײַװ פֿון יד ושם אין ירושלים אָפּגעפֿונען נאָך 130 זײַטן.

ב. די זײַטן װאָס פֿעלן האָבן בכן היפּשע בלויזן (אין רובֿ פאַלן פֿעלן  אין אַ צאָל ערטער נאָר אײנצלנע זײַטן, מיט פֿאַרצײכענונגען
פֿון אײנעם אָדער צװײ טעג, אָבער אין אַ קלײנער צאָל ערטער פֿאַלן עטלעכע צענדליג זײַטן כּסדר).
אײנע פֿון די װיכטיקסטע עובֿדות בײַם צוגרײטן דאָס „טעגבוך“ צום דרוק איז דעריבער געװען װי װײַט מעגלעך צו שאַפֿן, װי איך װאָלט עס אָנרופֿן, 
רינגען, װאָס זאָלן צונויפֿקײטלען די איבערגעריסענע כראָניק.

די ידיעות צום דערגאַנצן דאָס װאָס עס פֿעלט האָבן מיר גענומען פֿון אַרכיװ־מאַטעריאַלן (דאָקומענטן, גבֿית־עדותן וכדומה), פֿון אַ רײ שוין פֿאַרעפֿנטלעכטע
אַרבעטן װעגען װילנע און פֿון פּערזאָנען װאָס זענען אין יעדער צײַט געװען אין
װילנער געטאָ. בײַ יעדען פֿאַל האָבן מיר אָנגעװיזן דעם מקור.

די דאָזיקה אזוי גערופֿענע רינגען האָבן מיר געגעבן אויף די שײכדיקה ערטער מיט קלענער
שריפֿט און אין קאַנטיקה קלאַמערן, כּדי זײ זאָלן זיך בולט
אונטערשײדן פֿון קרוקס אָריגינעלן טעקסט. עס װערט דערבײַ אויך אָנגעגעבן
װיפֿל זײַטן פֿון אָריגינאַל עס פֿעלן אויף דעם געגעבענעם אָרט.

\endnumbering
\end{hebrew}
\end{RTL}
\end{Rightside}


\begin{Leftside}
\begin{english}
\section{
Introduction - Mordecai W. Bernstein \\  \RLE{
הקדמה ־ מרדכי װ. בערנשטײן
}.  }
\beginnumbering
\autopar

The story of Herman Kruk's \emph{Diary of the Vilna Ghetto}, specifically the reincarnation\footnoteA{
The term \RLE{גילגולים}
can also refer in biblical exegesis to the passage of the bones (literally tumbling over and over) of the deceased at the end of times through underground passages
to the Holy Land, where during the Messianic era the body will live again. Thus this Loshen Kodesh word, with its biblical overtones, captures the travails the book went through 
after being buried to finally being reconstituted and published.} 
of the aforementioned rare hurban-chronicle,
will be told by Pinkhos Schwartz in the biography of Herman Kruk which is provided further on. 
Here I only want to highlight a list of technical-editorial details.

1. The last page of the  (typewritten) original is numbered 757. However here and there in the middle a number of pages are missing,
and in the end we have in our control 510 pages of the original.
At first, in 1948, 380 pages were brought to New York. But, 11 years later in 1959, 
when we were already in the process of preparing the \emph{Diary} for publication, a further 130 pages were discovered in the archives of Yad Vashem in Yerushalayim.

2. The missing pages consequently are substantial gaps. (In most cases, in a number of places, only individual pages are missing - which would have recorded one or two days. 
However in a small number of places several tens of pages are missing in order). 
One of the most important issues in preparing the \emph{Diary} for publication is
therefore the consideration of how distant was the possibility to procure, what I like to call, \emph{links};
which would chain together the disconnected chronicle.

The information that supplements that which missing was taken from archive materials 
(documents, witness testimonies and the like), from a series of already published works
about Vilna, and from people which were in the Vilna at that time. In each case we have the
indicated the source.
 
The aforementioned, so called "links", we have provided in the suitable places in a smaller
script and surrounded by square brackets; in order that they should stand out clearly from 
Kruk's original text. 
These are thereby ``standing in the place of", in the relevant spots themselves, for the missing original pages. 


\endnumbering
\end{english}
\end{Leftside}

\end{pairs}
\Columns


\end{document}



















































