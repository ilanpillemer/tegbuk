\documentclass{article}
%\usepackage[margin=2in]{geometry}
\usepackage[osf,p]{libertinus}
\usepackage{microtype}
\usepackage[pdfusetitle,hidelinks]{hyperref}

\usepackage[series={A,B,C}]{reledmac}
\usepackage{reledpar}

\usepackage{graphicx}
\usepackage{polyglossia}
\setmainlanguage{english}
\setotherlanguage{hebrew}
\gappto\captionshebrew{\renewcommand\chaptername{קאַפּיטל}}
\usepackage{metalogo}


%%linenumincrement*{1}
%%\firstlinenum*{15}
%%\setlength{\Lcolwidth}{0.44\textwidth}
%%\setlength{\Rcolwidth}{0.44\textwidth}

\begin{document}
%%\maxhnotesA{0.8\textheight}
\renewcommand{\abstractname}{\vspace{-\baselineskip}}
\title{Diary of the Vilna Ghetto}
\author{Herman Kruk  \\ Transl. Ilan Pillemer}
\date{\today}

\maketitle

\tableofcontents
\newpage

\begin{pairs}

\begin{Rightside}

\begin{RTL}
\begin{hebrew}
\firstlinenumR{10000000}
\beginnumbering
%%\numberpstarttrue

\autopar

װעגען דער געשיכטע פֿון הערמאַן קרוקס „טאָגבוך פֿון װילנער געטאָ“, ספּעציעל װעגן די גילגולים פֿון דער דאָזיקער זעלטענער חורבן־כראָניק,
דערצײלט פּנחס שװאַרץ אין דער ביאָגראַפֿיאַ פֿון הערמאַן קרוקן װאָס מיר גיבן דאָ װײַטער.
דאָ װיל איך בלויז אָפּשטעלן אויף אַ רײ פּרטים פֿון טעכניש־רעדאַקציאָנעלן כאַראַקטער:

א. דאָס לעצע זײַטל פֿון אָריגינאַל (מאַשינשריפגט) טראַגט דעם נומער 757.
אָבער אין מיטן פֿעלן דאָ און דאָרט אַ צאָל בלעטער און אין סך־הכּל האָבן מיר אין אונדזער רשות 510 זײַטן פֿון אָריגינאַל.
תּחילת זענען אין 1948 אָנגעקומען קײן ניו־יאָרק 380 זײַטן, אָבער מיט 11 ײַאָר שפּעטער, אין 1959,
װען מיר האָבן שוין געהאַלטן אין מיטן צוגרײטן דאָס „טעגבאָך“ צום דרוק, האָבן זיך אין אַרכײַװ פֿון יד ושם אין ירושלים אָפּגעפֿונען נאָך 130 זײַטן.

ב. די זײַטן װאָס פֿעלן האָבן בכן היפּשע בלויזן (אין רובֿ פאַלן פֿעלן  אין אַ צאָל ערטער נאָר אײנצלנע זײַטן, מיט פֿאַרצײכענונגען
פֿון אײנעם אָדער צװײ טעג, אָבער אין אַ קלײנער צאָל ערטער פֿאַלן עטלעכע צענדליג זײַטן כּסדר).
אײנע פֿון די װיכטיקסטע עובֿדות בײַם צוגרײטן דאָס „טעגבוך“ צום דרוק איז דעריבער געװען װי װײַט מעגלעך צו שאַפֿן, װי איך װאָלט עס אָנרופֿן, 
רינגען, װאָס זאָלן צונויפֿקײטלען די איבערגעריסענע כראָניק.

די ידיעות צום דערגאַנצן דאָס װאָס עס פֿעלט האָבן מיר גענומען פֿון אַרכיװ־מאַטעריאַלן (דאָקומענטן, גבֿית־עדותן וכדומה), פֿון אַ רײ שוין פֿאַרעפֿנטלעכטע
אַרבעטן װעגען װילנע און פֿון פּערזאָנען װאָס זענען אין יעדער צײַט געװען אין
װילנער געטאָ. בײַ יעדען פֿאַל האָבן מיר אָנגעװיזן דעם מקור.

די דאָזיקה אזוי גערופֿענע רינגען האָבן מיר געגעבן אויף די שײכדיקה ערטער מיט קלענער
שריפֿט און אין קאַנטיקה קלאַמערן, כּדי זײ זאָלן זיך בולט
אונטערשײדן פֿון קרוקס אָריגינעלן טעקסט. עס װערט דערבײַ אויך אָנגעגעבן
װיפֿל זײַטן פֿון אָריגינאַל עס פֿעלן אויף דעם געגעבענעם אָרט.

ג. דאָס „טעגבאָך“ נעמט אַרום אַ פּעריאָד פֿון פֿולע צװײ יאָר װילנע אונטער
דער האַצישער אָקופּאַציע. די ערשטע דאַטע אין „טאָגביך“ איז דער 23סטער יוני
1941, די לעצטע - דער 14טער יולי 1943. אָבער דאָס איז זיכער נישט דער
סוף פֿון „טאָגבוך“. קודם־כּל פֿאַרענדיקט זיך די לעצטע זײַט אין מיטן פֿון אַ זאַץ,
און עס זענען פֿאַראַן ידיעות אַז הערמאַן קרוק האָט נאָך שפּעטער, סײַ אין
געטאָ סײַ װען ער שוין געװען אין אַ לאַגער אין עסטלאַנד, װײַטער געפֿירט
זײַן כראָניק און פֿאַרצײכנס די געשעענישן. אָבער דער סוף טאָגבוך איז צו
אונדז נישט דערגאַנגען

ד. עס איז נײטיק געװען צוצוגעבן הערות און אויסטײַשונגען פֿון אַ פֿאַרשײדענעם כאַראַקטער, למשל:


 (1) צוליב קאָנספּיראַטיװע טעמים האָט קרוק גאָר אָפֿט בײַם פֿאַרצײכענען
 נעמען אָנגעגעבן בלויז איניציאַלן אָדער נאָר דעם ערשטן נאָמען
 אָדער דעם פֿאַמיליע־נאָמען. האָט מען דאָס בכן געדאַרפֿט דעשיפֿרירן
 און דערגאַנצן. דאָס זעלביקע בנוגע אַ רײ אָרגאַניזאַציעס און אינסטיטוציעס װאָס זענען
 פֿאַרצײכנט מיט ראָשי־תּיבֿות אָדער קיצורים, װאָס
 זענען דעמאָלט, בימי געטאָס און לאַגערן, געװען באַקאַנט, אָבער הײַנט
 דאַרף מען זײ אויסטײַטשן. אַלע אַזעלכע דערגאַנצונגען האָבן מיר געגעבן
 אין קאַנטיקע קלאַמערן.

(2)
צו אַ סך נעמען האָבן מיר (אין באַזונדערע הערות) צוגעגעבן
ביאָגראַפֿישׂע פּרטים און װוּ מעגלעך אויך װעגן דעם װײַטערדיקן גורל
פֿון דער דערמאָנטער פּערזאָן, ספּעציעל װעגן עס רעדט  זיך װעגן מענטשן
װאָס האָבן פֿאַרנומען אַ געזעלשאַפֿטלעכע פּאָזיציע.

(3)
פֿון טײל זײַטן זענען אָפּגעריסן גאַנצע שטיקער, אָדער עס פֿאַלן
טײלן פֿון שורות (זען רעפּראָדוקציעס, זז' 270, 447), איז אויב מיר האָבן
זיך געקענט משער זײַן מיט אַ געװיסער מאָס זיצערקײט װאָס דאָרטן פֿעלט
האָבן מיר די דאָזיקע שטעלן דערגאַנצט און די דערגאַנצונגען אַרײַנגענומען
אין קאַנטיקע קלאַמערן. אויב מיר האָבן נישט געקענט עפּעס צוגעבן
האָבן מיר געשטעלט פּינטעלעך.

(4)
אין אַ צאָל ערטער, װוּ מיר האָבן באַמערקט אַ בפֿירושן טעות
אין טעקסט (למשל אין אַ דאַטע, אין אַ נאָמען אד“גל) האָבן מיר דאָס
אויסגעבעסערט אין אַ הערה.

(5)
גאַנץ אָפֿט באַגלײט קרוק זײַנע טאַג־גאַרצײכענונגען מיטן צוגאָב,
אַז ער לײגט בײַ אַ דאָקומענט, אָבער כּמעט אומטום איז דער
 דערמאָנטער דאָקומענט נישט בײַגעלײגט. טײל פֿון די דאָזיקע 
 דאָקומענט האָבן מיר געפֿונען אין אַרכיװ פֿון ייִװאָ (ס'רובֿ אין דער
  קאַטשערגינסקי־סוצקעװער־קאַלעקציע), אָדער ערגעץ אַנדערש װוּ, טײל זענען
  שוין אָפּגעדרוקט אין אַנדערע װערק. אין אַזעלכע פֿאַלן האָבן מיר אָדער
  רעפּראָדוצירט דעם דאָקומענט, אָדער אָנגעװיזן װוּ ער געפֿינט זיך.

ה. בדרך־כּלל האָבן מיר דער שׂפּראַך פֿוּן „טאָגבוזש“ גאָרנישט געביטן.
מיר האָבן איבערגעלאָזט אַזעלעכע און אויסדרוקן װאָס געװײנטלעך
װאָלט מען זײ אין אַ ייִװאָ־אויסגאַבע נישט געניצט. מיר האָבן אויך דאָ און
דאָרטן איבערגעלאָזט אַ גראַמאַטישן פֿעלער. געביטן האָבן מיר בלויז דעם
אויסלײג און אין געצײלט ערטער צוגעשטלעט אָדער איבערגעשטעלט אַ װאָרט
אויב דער זאַץ איז נישט געװען גענוג קלאָר. מיר האָבן אויך נישט מקפּיד
געװען אויף אײנהײטלעכקײט: אַ מאָל שרײַבט קרוק די געטאָ, און אַ מאָל
 דער געטאָ; אַ מאַל  װוּינונג־אָפּטײל און אַ מאָל װוּינונג־אָפּטײלונג אַד“גל.
 
 ו. מיר האָבן זיך באַמיט נישט צו באַשװערן דאָס בוך מיט צו פֿיל הערות.
 און װוּ עס איז נאָר מעגלעך געװען האָבן מיר געמאַכט די דערגאַנצונגען אין
 טעקסט גופֿה. דער עיקר האָבן מיר דאָס געטאָן בײַ נעמען פֿון פּערזאָנען און
 אינסטיטוציעס, למשל: [יעקבֿ] גערשטײן, [גערשון] פּלודערמאַכער, גרישע [יאשונסקי], [הערש] גוט[געשטאַלט], ג[ענס], „ב[ונד]“, „ר[רויטע“, אד“גל.
 
ז. װאָס שײך קאָנספּיראַציע איז „טאַגבוך“ נישט אין גאַנץן אויסגעהאַלטן.
אַפֿילו זיך פֿאַרצײכנט קרוק אונטער פֿאַרשײדענע נעמען און איניציאַלן:
הערמאַן, ה. קרוק, קר., פֿר' ק., אַד“גל. אזוי אַנדערע פּערזאַנען און אָרגאַניזאַציעס.

די טעכניש־רעדאַקציאָעעלע כּללים זענען באַשלאָסן געװען אויף אַ
באַראַטונג, װוּ עס האָבן זיך באַטײליקט: ישיה טרונק, ד"ר פֿיליפּ פֿרידמאַן ע"ה,
פּנחס שװאַרץ און דער שרײַבער פֿון די שורות.

עס איז נישט אונדזער אויפֿגאַבע צו פֿאַרגלײַכן דאָזיקה מיט
אַנדערע פֿאַרעפֿנטלעכטע אַרבעטן װעגן װילנער געטאָ װאָס זענען געשריבן
געװאָרן על־פּי זכּרון אָדער בלויז אויפֿן יסוד פֿון דאָקומענטן. מען קאָן אָבער
מיט זיכערקײט זאָגן, אַז דאָס „טעגבוך“ פֿון הערמאַן קרוק, װאָס איז געשריבן
געװאָרן אויף דער הײסער מינוט, תּיכּוף װי די געשעענישן זענען זיך פֿאַרלאָפֿן,
איז דער געטרײַסטער דאָקומענט פֿוּן יענער צײַט. זײער פֿיל פּרטים אין די
שוין פֿאַרעפֿנטלעכטע װערק װעגן װילנער געטאָ װעלן אין ליכט פֿון דאָזיקן
„טעגבוך“ מוזן רעװידירט װערן.

מרדכי װ. בערנשטײן

\endnumbering
\end{hebrew}
\end{RTL}
\end{Rightside}


\begin{Leftside}
\begin{english}
\section{
Introduction - Mordecai W. Bernstein \\  \RLE{
הקדמה ־ מרדכי װ. בערנשטײן
}.  }
\beginnumbering
\autopar

The story of Herman Kruk's \emph{Diary of the Vilna Ghetto}, specifically the reincarnation\footnoteA{
The term \RLE{גילגולים}
can also refer in biblical exegesis to the passage of the bones (literally tumbling over and over) of the deceased at the end of times through underground passages
to the Holy Land, where during the Messianic era the body will live again. Thus this Loshen Kodesh word, with its biblical overtones, captures the travails the book went through 
after being buried to finally being reconstituted and published.} 
of this rare hurban-chronicle,
will be told by Pinkhos Schwartz in his biography of Herman Kruk which is provided further on. 
Here I only want to highlight a list of technical-editorial details.

\emph{a}. The last page of the  (typewritten) original is numbered 757. However, in the middle a number of pages are missing here and there,
and in the end we have in our possession 510 pages of the original.
At first, in 1948, 380 pages were brought to New York. But, 11 years later in 1959, 
when we were already in the process of preparing the \emph{Diary} for publication, a further 130 pages were discovered in the archives of Yad Vashem in Yerushalayim.

\emph{b}. The missing pages consequently are substantial gaps. (In most cases, in a number of places, only individual pages are missing - which would have recorded one or two days. 
However several tens of pages are missing consecutively in a small number of places). 
One of the most important issues in preparing the \emph{Diary} for publication was
therefore the consideration of how remote was the possibility to create, what I like to call, \emph{links};
which would chain together the disconnected chronicle.

The information that supplements that which missing was taken from archive materials 
(documents, witness testimonies and the like), from a series of already published works
about Vilna, and from people who were in the Vilna at that time. In each case we have the
indicated the source.
 
These, so called "links", we have provided in the suitable places in a smaller
script and surrounded by square brackets; in order that they should stand out clearly from 
Kruk's original text. 
These are thereby ``standing in the place of" the missing original pages, in the relevant spots themselves.

\emph{c}. The \emph{Diary} covers a period of two full years of Vilna under the Nazi occupation.
The first date in \emph{Diary} is the 23\textsuperscript{rd} June 1941, the last - the 14\textsuperscript{th}
July 1943. However, this is definitely not the end of \emph{Diary}. First of all, the 
last page ends in the middle of a sentence, and there is also evidence that
Herman Kruk continued later, both in the Ghetto and also when he was in a camp in Estonia,
to further manage his chronicle and record what was taking place. But the rest of the diary
did not make its way to us.

\emph{d}. It was necessary to provide annotations or explications of varied character, for example:


(1) On account of conspiracy reasons Kruk often completely tokenised names, substituting
only initials, or just the first name or surname. These need to therefore be deciphered and
supplemented. There is the same concern with a series of organisations and institutions which
are tokenised with acronyms or abbreviations - which were at the time, in the days of the
ghetto and camps, known. However, today, these need to be explicated. All of these supplements we
have given in square brackets.

(2) For many names we have (through separate annotations) provided biographical details
and where possible also the eventual fate of the referenced person, especially when the people
who are being spoken about took up a communal position.

(3) From a few pages whole pieces have been torn off, or sections of lines are missing, (see
facsimiles on pages 270, 447). Therefore if we were able to surmise with a reasonable
degree of confidence what was missing, we provided supplements and surrounded the supplements
with square brackets. If we did not know what to provide we then inserted a set of dots.

(4) In a number of places where we noticed obvious mistakes (for example in a date, in a name, and the similar),
we corrected them in an annotation.

(5) Very often Kruk linked his diary notes to an addendum, such as referring to a document, but
almost all of these mentioned documents aren't provided. Some of these documents we have
found in the YIVO archives (mostly in the Sutzkever Kaczerginski Collection\footnoteA{http://www.yivoarchives.org/index.php?p=collections/findingaid\&id=932702\&q=Sutzkever\&rootcontentid=219617}),
or somewhere else, some were already published in other work. In such cases we either reproduce the document or detail where it can be found.

\emph{e}. Normally we have kept the language of \emph{Diary} as constructed. We have left alone such words
and expressions which normally one would not use in a YIVO publication. We have also left alone here
and there grammatical mistakes. Thus we have only altered the orthography and in rare places added or moved a word
when otherwise the sentence would not be sufficiently clear. We have not been fastidious with uniformity: sometimes Kruk
writes \emph{the} ghetto  with a feminine article and sometimes with a male article; sometimes he writes the word \emph{section}
spelt one way and sometimes spelt another way, and other similar examples. 

\emph{f}. We have made an effort to not burden the book with too many annotations.
And where it was possible we have made the additions in the text itself. We have
in particular done this with names of people and institutions. For example:
[Yankel] Gerstein, [Gershon] Pludermacher, Grisha [Yashunski], [Hersh] Gut[gestalt], G[ens],
B[und], R[oiter], and other similar examples.

\emph{g}. What is considered conspiratorial in \emph{Diary} is not quite consistent.
Even Kruk himself is notated with different names and initials: Herman, H. Kruk,
Kr., Mr K and more similar examples. And so with other people and organisations.

The technical editorial rules were decided through consultation; wherein participated:
Isaiah Trunk, Dr Philip Friedman (may he rest in peace), Pinchas Schwartz and the
writer of these lines. 

It is not our task to compare this \emph{Diary} with other published works about the Vilna ghetto,
which have been written from remembrances or only on the basis of documentation. One can say with certainty
that this \emph{Diary} from Herman Kruk which was written in the heat of the moment, in the moment as the events themselves
were unfolding; is the most faithful document from that period. So many details in already published works about the Vilna
ghetto will need to be revised in the light of this \emph{Diary}.

Mordecai W. Bernstein

\endnumbering
\end{english}
\end{Leftside}

\end{pairs}
\Columns

\newpage

\begin{pairs}

\begin{Rightside}

\begin{RTL}
\begin{hebrew}
\firstlinenumR{10000000}
\beginnumbering
%%\numberpstarttrue

\autopar

הערש (הערמאַן) קרוּק\\  (19טער מײַ 1897 - 19טער (?) סעפּטעמבער 1944)\\

זײַן לעבן, זײַן טעטיקײט אין װילנער געטאָ און זײַן אומקוּם\\ 

דער זומער פֿון יאָר 1944 איז געװען פֿאַר די היטלער־אַרמײען אין אײראָפּע
אַ פּעריאָד פֿון װיסטע מפּלות אויף אַלע פֿראָנטן.אומעטום זענען די דײַטשע
אַרמײען געטריבן געװאָרן אויף צוריק. סאָװעט־רוסלאַנד איז שוין געװען כּמעט
אין גאַנצן אָפּגערײניקט פֿון דער נאַצישער אָקופּאַציע. אויף צפֿון איז לענינגראַד
שוין געװען באַפֿרײַט פֿון דער לאַנגדויערנדיקער באַלאַגערונג און די דײַטשע
אַרמײען זענען שוין געהאַט אַנטלאָפֿן קײן נאָרװעגיע אוּן עסטלאַנד. עס האָט זיך
אָנגעהויבן די באַפֿרײַונג פֿון די באַלטישע לענדער.

דעם 19טן סעפּטעמבער 1944 האָבן סאָװעטישע פֿאָרפּאָסטנס דערגרײכט
די באַנסטאַציע קלאַג, נישט װײַט פֿון דער עסטלענדישער הויפּטשטאָט טאַלין
(רעװעל). נאָכיאָגנדיק די אַנטלויפֿנדיקע היטלערישער אַרמײ־אָפּטײלן האָבן די
סאָװעטישע טאַנקלײַט דערזען נישט װײַט פֿון שאָסײ גרויסע ברענענדיקע שײַטער־הויפֿנס.
קנוילן רויך האָבן זיך געטראָגן צו די הימעלען און די לופֿט איז
געװען דורצגעזאַפּט מיטן ריח פֿון צעברענטן פֿלײש. אַ געװיסע צאָל ציװילע
מענער און פֿרוען האָבן אין פּאַניק זיך באַװוּיגן אַהין און צוריק לענג אַויס דעמ װאַלד.

אַ סאָװעטישער קריגס־קאָרעספּאָנדענט גיט װעגן דעם איבער די װײַטערדיקע אימהדיקע פּרטים:
\emph{
„עס איז אוממעגלעך איבערצוגעבן מיט װװערטער די געפֿילן פֿון די
סאָװעטישע קריגסלײַט, װען זײ האָבן זיך דערװוּסט, אַן אויף די דערנעבדיקע
שײַטער־הויפֿנס בראָטן זיך הרוגים, פֿרידלעכע מענטשן, װאָס
זענען אויסגעמאָרדט געװאָרן דורך די דײַטשן ־ עסטלענדער, לעטן און
ליטװינער, װאָס זענען געהאַלטן געװאָרן אין קאָנצענטראַציע־לאַנגער בײַ
דער סטאַציע קלאָגע. די דײַטשן האָבן נישט באַװיזן צו פֿאַרװישן די
שפּורן. די שײַטער־הויגנס און די גבית־עדותן פֿון געראַמטעװעטע (דערװײַל
האָט מען באַװיזן אָנצוצײלן בערך 100 אָפּגעראַטעװעטע) גיבן
אַ מעגלעכקײט צו רעסטאַװרירן דאָס בילד פֿון דעם נײַעם מאוימדיק.
היטלערישן פֿאַרברעכן.“}\footnoteA{\RLE{
י. אָסיפּאָװ אין „איזװעסטיאַ“, נומ' 231, מאָסקװע, 28סטן סעפּטעמבער 1944.
}}


די אויסגעמאָרדטע „עסטלענדער, לעטן און ליטװינער“ זענען ס'רובֿ געװען
ייִדן פֿון װילנע, קאָװנע און פֿון אַ רײ באַלטישע שטעט. אין מאָמענט פֿוּן דער
דאָזיקער שוידערלעכער הריגה זענען געװען אין לאַנגער בערך 2,000 ייִדן. עס
זענען דעמאָלט דאָרט אויך געװען בערך 1,000 סאָװעטישע קריגס־געפֿאַנגענע און
עסטלענדער װאָס זענען אַרײַנגעטריבן געװאָרן אין דאָזיקן לאַגער אַרײַן.
צוזאַמען איז בעת דער הריגה אַזוי אַרום געװען אין קלאָגע און אין די אַרומיקע
סובלאַגערן בערך 3,000 נפֿשות ־ מענער, פֿרויען און קינדער.

דער דערמאָנטער סאָװעטישער קריגס־קאָרעספּאָנדענט גיט איבער אין
דער ציטירטער קאָרעספּאָנדעקע, אַז אַ רײ געראַטעװעטע האָבן אים גענוי דערצײלט
װאָס ס'איז זיך פֿאַרלאָפֿן יענעם טאָג אין קלאָגע. ער רעכנט אויס די נעמען
פֿון די געראַטעװעטע, װאָס האָבן צוגעשטעלט די אינפֿאָרמאַציע פֿאַר זײַן קאָרעספּאָנדענץ:
 ד“ר בוזשאַנסקי, פּראָװיזאָר באַלבערישעסקי, אינזשעניר ראַטנער,
 אַדװאָקאַט אָלײסקי און דער בוכהאַלטער אַנאָמיק (אַנאָליק?). 
נאָר װעגן אײנעם פֿוּן די דאָזיקע אָפּגעראַטעװעטע (אַדװאָקאַט אַלײסקי) זאָגט דער סאָװעטישער
צײַטונג־קאָרעספּאָנדענט בפֿירוש, אַז ער איז אַ װילנער. פֿוּן אַנדערע קװאַלן
װײסן מיר, אַז אויך די איבעריקע אויסגערעכנטע זענען װילנער.

די אָפּגעראַטעװעטע דערצײלן (דאָס װערט גענוי באַשריבן איִן דער ציטירטער
 קאָרעספּאָנדענץ) װי די אַנטלויפֿנדיקע דײַטשע מיליטערישע אָפּטײלן
 האָבן שטראָמענװײַז זיך געטראָגן פֿאַרבײַ קלאָגע. װי מען דאַרף דרינגען פֿון
דער קאָרעספּאָנענץ, האָט דאָס אַזוי געדויערט מער װי אײן טאָג. דער סטראָם
אַנטלויפּנדיקע דײַטשע מיליטער־אָפּטײלן איז געװאָרן אַלץ שטראַרקער און שטאַרקער.
די דײַטשע לאַגער־װעכטער האָבן אָנגעזאָגט די אַרעסטאַנטן צו זײַן גרײט
עװאַקויִרט צו װערן. אין לאַגער האָט זיך באַװיזן אַ „זאָנדערקאָמאַנדעע“ פֿון
עס־עס־לײַט, װאָס איז אָנגעקומען פֿון דער נישט־װײַטער עסטלענדישער
הויפּשטאַט טאַלין.

דעמ 19טן סעפּטעמבער, 5 פֿאַר טאָג, האָבן די עס־עס אַרויסגעטריבן אַלעמען
פֿון די באַראַקן. צװישן די אַרעסטאַנטן האָבן זיך געפֿונען עטלעכע מעוברטע
פֿרויען. בײַ אַײנער פֿון זײ האָט זיך גראָד געהאַט אָנגעהויבן די קימפּעט.

געװײנטלעך זענען די לאַגערניקעס געװען פֿאַרנומען מיט בעטאָן־אויסאַרבעטונג.
אָבער דעם פֿרימאָרגן האָבן די דײַטשן זיך ענערגיש פֿאַרנומען מיט
אָנקלײַבן געהילץ. יעדער פֿון די אַרעטאַנטן האָט געדאַרפֿט ברענגען פֿון װאַלד
אַ צװײמעטערדיקן קלאָץ פֿון סאָסנעהאָלץ און עס אַװעקלײגן אויף דער „פּאָליאַנע“
לעבן לאַגער. מען האָט באַפֿוילן די אַרעסטאַנטן אויסצולײגן די  קלעצער אויפֿן
גראָז אינ אַ רײ. דערנאָך האָט מען אָפּגעצײלט אַ געװיסע צאָל אַרעסטאַנטן
און זײ באַפֿוילן זיך אויסצולײגן אויף די קלעצער מיטן פּנים צו דער ערד, האַרט 
אײנער לעבן  צװײטן ־ אזוי אַז זאָלן פֿאַרנעמען װאָס װײניקער פּלאַץ. דעמאָלט 
האָבן עטלעכע עס־עס־לײַט מיט מאַשינביקסן גענומען שיסן אויף די אויסגעלײגט.
דער געהילך פֿון די שאָסן האָט נישט געקענט דערשטיקן די קרעכצן
פֿון די אויסגעמאָרדטע. דאָ און דאָרט האָט אײנער פֿון די קרבנות נאָך באַװיזן זיך
אַ ריס צו טאָן פֿון דער ערד און נעמען לויפֿן, באַגיסנדיק זיך מיט בלוט. אַזאַ
אײנעם האָבן די דײַטשן נאָכגעשאָסן און צוריקגעשלעפּט צו די קלעצער.

װי נאָר זײ זענען „פֿאַרטיק געװאָרן“ מיט דער ערשטער גרופּע קרבנות
האָבן די עס־עס־לײַט אָפּגעקליבן אַ צװײטע גרופּע און זײ געצװוּנגען אויסצולײגן
קלעצער אויף די דערסאָסענע און זיך אַלײן אויסלײגן אויף די קלעצער. װידער
האָבן גענומען אַרבעטן די מאַשינביקסן און װידער האָט מען צוגעפֿירט צו
דער עקידה אַ װױטערדיקע גרופּע.

װען עס איז שוין געהאַט אָנגעװאַקסן אַ באַרג אויסגעמאָרדטע פֿון 4־5 רײען
הויך, האָבן די עס־עס־לײַט גענומען אויסלײגן אויפֿן גראָז אַ נײַע רײ קלעצער.
אַרום האַלבן טאָג זענען אויפֿן פֿעלד לעבן דעם לאַגער שוין געװען אָנגעלײגט
4 אַזעלכע בערג מיט הרוגים. אָבער װײַט נישט אַלע אַרעסטאַנטן זענען שוין
געװען אויסגעשאָסן.

\endnumbering
\end{hebrew}
\end{RTL}
\end{Rightside}


\begin{Leftside}
\begin{english}
\section{
Biography of Herman Kruk - Pinchas Schwartz. \\  \RLE{
ביאָגאַפֿיע פֿון הערמאַן קרוק - פּנחס שװאַרץ.
}  }
\beginnumbering
\autopar

Hersch (Herman) Kruk \\ \\ (19th May 1897 - 19th (?) September 1944) \\ 

His life, his activity in the Vilna Ghetto, and his death. \\

The summer of 1944 was for the Hitler-Armies in Europe a period
of dismal failures on all fronts. Everywhere the German armies 
were being driven back. Soviet Russia was already almost completely
clear from the Nazi occupation. In the north Leningrad was already
liberated from the long-lived siege and the German armies had already
fled to Norway and Estonia. It was the beginning of the liberation of the Baltic
lands.

On the 19th September 1944 the Soviet outmost positions reached the Klooga train station,
not far from the capital Talinn (Reval). Pursuing after the fleeing army divisions of Hitler, the
Soviet tank forces caught sight of large burning bonfires not far from the paved road.  Plumes
of smoke reached the heavens and the air was saturated with the smell of burning flesh. Groups of numbers of
men and women civilians where running there and back along the length of the forest.

A Soviet war-correspondent gave the following horrible details about the scene:
\emph{``It is impossible to convey in words the feelings of the Soviet armed forces when they
became aware of it, that the approaching bonfires were burning the murdered, innocent people who
had been put to death by the Germans - Estonians, Latvians and Lithuanians who had been held in
concentration camps at the Klooga station. The Germans did not succeed in removing their tracks.
The bonfires and the witness-testimonies of survivors 
(at this time we have been able to count about 100 survivors) give a possibility to restore the picture
of this new ugly Hitlerian crime."}

The murdered ``Estonians, Latvians and Lithuanians" were mostly Jews from Vilna, Kovne and a series
of Baltic towns. In the moment of that horrific murder there were in the camps about 2,000 Jews.
At that time there were also about 1,000 Soviet prisoners-of-war and Estonians who had been forced
into these camps. Altogether at the time of the murder there were around 3,000 souls in Klooga and in the 
surrounding sub-camps - men, women and children.

The aforementioned Soviet war correspondent writes in his referenced correspondence that a 
series of survivors precisely narrated to him that which had unfolded that day in Klooga. He lists the names
of the survivors whom provided the information for his correspondence: Dr Buszanski, Provider Balaberishesky,
Engineer Ratner, Advocate Aliysky, and the accountant Anomik (Anolik?). Only one of these survivors (Advocate
Aliysky) says to the Soviet newspaper correspondent explicitly that he is a Vilnian. From other sources we know
that also the rest of the listed people were Vilnians.

The survivors said (this is precisely described in the quoted correspondence) that the fleeing German military
divisions had been moving past Klooga in a stream. One can infer from the correspondence that this continued 
for longer than one day. The stream of fleeing German military divisions had been becoming still stronger and
stronger. The German camp guards told the prisoners to get ready to be evacuated. A "Sonderkommando"
from the S.S. forces, who had arrived from the nearby Estonian capital Talinn, had shown up at the camp.

On the 19th September, 5 a.m., the S.S drove everyone out from the baracks. Amongst the prisoners there could be found
a few pregnant women. One of those women had just then gone into labour.

Normally the camp inmates were busy with the manufacture of concrete. But that morning the Germans had them frenetically
busy with collecting lumber. Each one of the prisoners was required to bring from the forest a two-meter-thick log of pine wood and
lay it out on the \emph{Polione} next to the camp. The prisoners were charged to lie the logs out on the grass in a row. After this,
a certain number of prisoners were counted off and ordered to lie themselves out on the logs with their faces towards the earth,
one very close to the next - in such a way that they should use a little space. At this point a few S.S forces began to shoot, with
machine guns, upon the laid out prisoners. The sound of the shots did not quiet out the choking of the murdered. Here and there
one of the victims managed to jump up from the ground and begin to run, covered in blood. Then together the Germans shot after him 
and he was dragged back to the logs. 

As soon as they were "finished" with the initial group of victims
the S.S forces selected a second group and they were directed to lay out logs on those who had been shot and then
to lie themselves out on the logs. Again the machine guns began to work and again they directed a further group to the
pyre.

When a mountain of the murdered had already grown to 4/5 rows high, the Germans began to lie out on the grass
a new row of logs. At approximately midday, on the field next to the camp there was already 4 different mountains
of executed prisoners. But far from all the prisoners had already been shot.

\endnumbering
\end{english}
\end{Leftside}

\end{pairs}
\Columns


\end{document}



















































